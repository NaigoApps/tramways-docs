\chapter{Servizi di cloud computing}\label{ch:chapter1}

\section{Software as a service}
Per \textit{software as a service} si intendono un'insieme di applicazioni accessibili al cliente tramite Internet e spesso semplicemente attraverso un browser.\\
L'utente non possiede il software ma paga per poterlo utilizzare, solitamente tramite un abbonamento o un costo a consumo.\\
I servizi sono ospitati su una piattaforma remota gestita da un provider.

\section{Platform as a service}
Per ospitare un \textit{SaaS} è possibile servirsi di una \textit{Platform as a Service}.\\
Anche in questo caso si tratta di un servizio accessibile tramite Internet, ma quella che viene offerta è una vera e propria piattaforma che fornisce un framework su cui sviluppare e caricare applicazioni.\\
Una \textit{PaaS} è rivolta agli sviluppatori, e prevedono anche in questo caso piani di pagamento a consumo o abbonamenti.
Uno dei vantaggi principali è che l'infrastruttura sottostante è condivisa con altri utenti, cosa che limita il costo del servizio.

\section{Infrastructure as a service}
Quando si necessita di un maggior controllo e flessibilità è possibile affidarsi ad una soluzione \textit{Infrastructure as a service}.
In questo caso il cliente può arrivare a gestire l'intero sistema operativo senza comunque preoccuparsi della livello hardware sottostante.