\chapter{Architetture}

\section{Introduzione}

Il cuore di un software object-oriented si trova nel modello di dominio. Questo raccoglie i concetti estratti dall'analisi dei requisiti ed è composto da una rete di oggetti dotati di attributi e metodi che rispecchiano i \textit{dati} da memorizzare e le \textit{operazioni} necessarie alla loro elaborazione.
Questi elementi non dovrebbero avere al loro interno codice di supporto (finalizzato alla memorizzazione su un database, alla visualizzazione su interfaccia, all'invio su un canale ecc...), e in oltre questo non dovrebbe contenere frammenti di logica di dominio.
Il rischio altrimenti è quello di abbassare in maniera critica la manutenibilità del codice: le varie componenti sono accoppiate tra loro, e devono evolvere necessariamente in contemporanea.
Il test automatico diventa complesso, e si rischia di raggiungere presto lo stato di \textit{big ball of mud}\cite{microservices_architecture}.
Scrivere codice il più possibile disaccoppiato dalle tecnologie di contorno lo rende più manutenibile e testabile, ma d'altra parte può diminuire la produttività del team di sviluppo, soprattutto quando l'applicazione da sviluppare non è di eccessiva complessità.

Nel caso del lavoro corrente l'obiettivo è simulare l'evoluzione di un software da una versione monolitica ad una a microservizi: per far questo sono state sperimentate alcune architetture riconosciute, applicando quando possibile il principio di separazione delle responsabilità, fondamentale quando si parla di microservizi.

La prima architettura presa in esame è quella a livelli.

\section{Architettura a livelli}

Questo tipo di architettura è uno degli standard più affermati nello sviluppo di applicazioni enterprise.\\
Il codice è organizzato a livelli sovrapposti, ognuno dei quali può sfruttare i servizi esposti \textbf{soltanto} dai quelli sottostanti.
Esistono una serie di layers standard, di seguito se ne riportano alcuni dei più comuni\cite{ddd}:
\begin{itemize}
	\item Presentation: Fornisce informazioni e interpreta i comandi provenienti dall'esterno, in modo da interfacciarsi con un utente o un'altro software.
	\item Application: Sottile strato che dirige la logica di dominio delegando le azioni ai business objects.
	\item Model: Mantiene lo stato della logica di dominio e contiene le regole che la governano. \'E il cuore del software.
	\item Infrastructure: Fornisce funzionalità utili agli altri strati come persistenza, invio di messaggi, creazione interfaccia grafica ecc...
\end{itemize}
In questa visione il livello di business logic è consapevole dei livelli sottostanti, e può interagire direttamente con essi.
Dovendo gestire più ad alto livello la logica di dominio, per esempio aprendo e chiudendo una transazione, anche il livello application necessita di interagire con lo strato di infrastruttura.

Le linee guida per realizzare un'applicazione seguendo questa architettura prevedono di partizionare il codice in livelli coesi e dipendenti solamente da quelli sottostanti.

Esistono varie versioni di questa architettura, in cui i livelli di contorno variano: una di queste è la \textit{three layered architecture} (vedi figura \ref{fig:layered-architecture}), che prevede i livelli di \textit{Presentation} per la gestione della UI, \textit{Business Logic} per la logica di dominio e \textit{Persistence} per dialogare con i database.
Questi sono sufficienti per gran parte della applicazioni enterprise, in particolare se l'approccio usato è quello del monolite o del monolite a servizi.\\

\begin{figure}[h]
	\centering
	\includegraphics[width=\textwidth]{img/layered-architecture}
	\caption{Architettura a tre livelli}
	\label{fig:layered-architecture}
\end{figure}

Ogni livello può avere più istanze o essere ulteriormente separato, per esempio quando vi sono più interfacce grafiche verso la stessa applicazione, o più database che forniscono il servizio di persistenza.\\
Gli oggetti del modello di dominio devono essere, come già affermato, indipendenti dalle varie rappresentazioni, siano queste destinate ad UI, persistenza o altro.
Un indice di cattiva separazione delle responsabilità può essere per esempio la necessità di duplicare codice, o di modificare la logica di dominio quando si aggiunge una nuova interfaccia all'applicazione.\\
La separazione in layers permette inoltre di poter effettuare il rilascio delle varie parti dell'applicazione su macchine differenti, favorendone l'evoluzione e la scalabilità.
\section{Ports and adapters}

Una delle alternative all'architettura organizzata a \textit{layers} è l'architettura chiamata \textit{ports and adapters} o \textit{hexagonal architecture}\cite{microservices_architecture}.

\begin{figure}[h]
	\centering
	\includegraphics[width=\textwidth]{img/hexagonal-component-diagram}
	\caption{Architettura ports and adapters}
	\label{fig:hexagonal-architecture}
\end{figure}

Questo modo di organizzare i componenti del sistema è focalizzato sull'isolamento della logica di dominio.\\
Le comunicazioni di input/output con l'esterno sono guidate da interfacce \textbf{definite dal nucleo} e chiamate \textit{porte}: queste possono essere di tipo \textit{inbound} quando un modulo esterno ha necessità di invocare la logica di dominio, o di tipo \textit{outbound} quando il nucleo ha la necessità di invocare servizi esterni.
Ogni porta può avere più \textit{adattatori} associati, elementi che servono per collegare realmente i componenti esterni al nucleo.
La logica di dominio implementa le porte di tipo \textit{inbound}, che saranno usate dagli adattatori in ingresso per richiedere servizi all'applicativo.
Gli adattatori in uscita implementano le porte di tipo \textit{outbound}.

Partendo da un sistema costruito a livelli è possibile derivare l'equivalente in architettura esagonale considerando i seguenti fatti:
\begin{itemize}
	\item Se il client di un'applicazione a livelli è un'altro software, il livello di presentazione del primo può essere molto legato al livello di persistenza del secondo: ciò che è esterno alla logica di dominio quindi puòò essere visto semplicemente come un'interfaccia esterna
	\item In un'architettura a livelli moderna difficilmente la progettazione inizia dal livello di persistenza: spesso questo è realizzato da classi DAO (Data Access Object) che implementano un'interfaccia definita sulla base delle necessità del livello di business logic: questo crea una dipendenza in qualche modo inversa rispetto alla filosofia dell'architettura, enfatizzando il fatto che la logica di dominio è il centro dell'applicativo.
\end{itemize}

Rifattorizzare un progetto basato su architettura three tier per renderlo compatibile con quella ports/adapters a volte potrebbe limitarsi alla riorganizzazione dei package, senza necessitare modifiche al codice: questo perché i componenti di un'architettura a livelli spesso hanno un equivalente in architettura esagonale (vedi figura ).

Può capitare che data un'applicazione enterprise moderna
Anche in questo caso si cerca di rendere il più possibile indipendente il nucleo di un'applicazione dalle tecnologie di contorno, come database, sistema di scambio messaggi ecc...

Convertire un'applicazione service oriented realizzata a layers in una aderente all'architettura esagonale non è un'operazione complessa: se per esempio lo stack è di tipo 3-tier con una serie di servizi REST esposti che invocano internamente la logica di dominio ed uno strato di persistenza (es: JPA), il mapping è immediato (vedi figura \ref{fig:layered-hexagonal}).

\begin{figure}[h]
	\centering
	\includegraphics[width=\textwidth]{img/layered-hexagonal}
	\caption{Confronto architettura a livelli e ports/adapters}
	\label{fig:layered-hexagonal}
\end{figure}

I benefici di tale organizzazione diventano evidenti quando utilizzata all'interno di un software a microservizi, in quanto si è obbligati a identificare con chiarezza quali sono i confini del servizio, favorendo manutenibilità e testabilità.

\section{Logica di dominio}

Quando si parla di realizzare la logica di dominio per un'applicazione enterprise vi sono numerosi approcci che si possono adottare: di seguito riportiamo alcuni patterns  notevolu evidenziando quali sono i punti di forza e le debolezze di ognuno.

\subsection{Transaction script}
Il pattern Transaction Script\cite{enterprise_app} prevede che la logica di dominio sia organizzata in base alle transazioni necessarie al sistema.
Per ognuna di esse vi sarà un frammento di codice procedurale che si occupa di recuperare i dati necessari all'elaborazione, effettuare validazioni, modifiche ed aggiornare il database.
Questo approccio è particolarmente adatto per casi d'uso in cui la logica è semplice: si tratta di organizzare correttamente il codice in moduli evitando duplicazione.
In riferimento alla figura \ref{fig:layered-hexagonal}, i componenti \textit{Business Service} sono in effetti le classi \textit{Transaction Script}, prive di stato e ricchie di logica; i componenti \textit{Model} sono classi di modello che contengono lo stato vero e proprio, ma povere di comportamento (es. POJO in Java).\\
Uno dei casi in cui questo approccio può essere utile, è quando si utilizza un framework come JPA per la persistenza: si ha infatti il vantaggio di ridurre drasticamente il numero di righe di codice di tale livello, accettando il compromesso di "sporcare" le classi di modello con alcune annotazioni accoppiate ad una tecnologia.
Rendendo la separazione tra livelli (o tra nucleo e porte) più labile, questo pattern è adatto quando la logica di dominio non è particolarmente complessa e si prevede che non evolverà molto.

\subsection{Domain model}
Il pattern Domain Model\cite{enterprise_app}, al contrario di Transaction Script, organizza la logica di dominio all'interno degli oggetti del modello: \textit{Model} in figura \ref{fig:layered-hexagonal}.
Le classi della categoria \textit{Business Service} sono presenti e presentano come in \textit{Transaction Script} un metodo per ogni richiesta/transazione, ma in questo caso la maggior parte del comportamento viene delegato.

In un'applicazione Object Oriented, gli oggetti cercano di modellare dati e comportamenti caratteristici dell'area di business di cui tratta il software.
Quando si ha a che fare con un modello ad oggetti ricco, forzarlo a rispecchiare semplicemente tabelle di un database vanifica i benefici della coesione tra dati ed operazioni forniti dalla OOP.
Senza contare il fatto che non esiste sempre una corrispondenza esatta tra le associazioni possibili tra oggetti e quelle tra entità memorizzate su database (basti pensare al tema dell'Object Relational Mapping o ai Design Patterns\cite{design_patterns}).\\
Utilizzare un vero e proprio modello ad oggetti permette di avere codice più comprensibile e manutenibile, preferendo numerose piccole classi ognuna con le proprie responsabilità.

Quando si parla di architettura a microservizi anche quest'ultimo approccio può mostrare alcuni punti di debolezza.
Può essere utile effettuare un ulteriore raffinamento, arrivando al concetto di Domain-driven design.

\subsection{Domain-driven design}
Questo raffinamento dello sviluppo orientato agli oggetti\cite{ddd}\cite{microservices} è stato pensato per applicazioni dotate di logica di business complessa.\\

DDD prevede alcuni patterns utilizzabili come elementi base per costruire un modello di dominio.
Di seguito i principali:
\begin{itemize}
	\item \textit{Entity:} Oggetto che ha un'identità persistente identificate da un ID. Sono le classi che in JPA sono annotate come \textit{@Entity}.
	\item \textit{Value object:} Collezione di valori priva di un'identità persistente. Non possiedono un'ID, quindi due istanze diverse con lo stesso stato interno possono essere usate in modo intercambiabile.
	\item \textit{Factory:} Oggetto o metodo che si occupa di creare oggetti complessi.
	\item \textit{Repository:} Oggetto che permette l'accesso alle entità persistenti e nasconde i meccanismi di accesso al database.
	\item \textit{Service:} Oggetto che implementa la logica di business che non è interna alle entities e ai value objects.
\end{itemize}

Un altro concetto importante in questo contesto è quello di \textit{Aggregato}.

\subsection{Aggregati e microservizi}
In un'architettura a microservizi ogni servizio deve necessariamente avere il proprio modello di dominio.
Un tema interessante che verrà approfondito in questo paragrafo è quello del partizionamento di un modello di business in \textit{sottodomini} utilizzando il concetto di \textit{Business Context}.

Un diagramma delle classi può non essere esaustivo quando si parla di confini di oggetti di business.
Solitamente una transazione coinvolge un oggetto principale, ma può spaziare anche su altri elementi secondari.
Un oggetto di business solitamente ha delle invarianti, ovvero condizioni che devono essere vere in ogni momento; in un'applicazione multiutente con entità persistite per esempio non è semplice studiare le interazioni con il database in modo da evitare inconsistenze.\\
Modellare un modello di dominio utilizzando aggregati significa avere una serie di grafi di oggetti che possono essere trattati come un'unità.
In questo modo le operazioni diventano più chiare: le operazioni di ricerca, aggiornamento e cancellazione per esempio coinvolgono un aggregato nella sua interezza, permettendo di evitare complicazioni come:
\begin{itemize}
	\item Lazy loading: Identificare gli oggetti da caricare diventa immediato.
	\item Eliminazione: Non esisteranno mai oggetti orfani sul database: l'eliminazione verrà sempre fatta in blocco.
	\item Concorrenza: Quando si effettua un'operazione su un elemento di un aggregato è possibile utilizzare un locking sulla sua radice, evitando inconsistenze.
\end{itemize}

Per godere dei benefici portati dagli aggregati è necessario rispettare alcune regole apparentemente restrittive:

\begin{enumerate}
	\item Le operazioni sull'aggregato devono partire dalla radice. Non dev'essere quindi possibile aggiornare le sue componenti direttamente. Una volta ottenuto l'oggetto principale tramite un repository basterà chiamare alcuni suoi metodi per effettuare le modifiche necessarie.
	\item Gli aggregati devono riferirsi tra loro con una chiave primaria. Non devono quindi essere utilizzati oggetti di tipo riferimento quando si punta ad un aggregato diverso.
	Questo è un approccio diverso da quello che solitamente si adotta in OOP, ma questo permette di disaccoppiare gli aggregati, rende più definiti i confini evitando modifiche non intenzionali e permette rende la separazione in servizi molto più agevole.
	\'E inoltre più semplice scalare i database in maniera verticale, ovvero separandolo per aggregato.
	\item Ogni transazione crea o aggiorna uno e un solo aggregato.
	Nel caso un'operazione di business debba operare con più aggregati sarà quindi necessario creare più transazioni locali.
\end{enumerate}

In un'applicazione monolitica questi limiti possono sembrare stringenti, ma ragionando in un'ottica \textit{microservices oriented} si noterà che molti problemi in questo modo risulteranno risolti.